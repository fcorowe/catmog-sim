% Options for packages loaded elsewhere
\PassOptionsToPackage{unicode}{hyperref}
\PassOptionsToPackage{hyphens}{url}
\PassOptionsToPackage{dvipsnames,svgnames,x11names}{xcolor}
%

\documentclass[
  a4paper, 
  twoside,
  final
]{article}

\usepackage{amsmath,amssymb}
\usepackage{iftex}
\ifPDFTeX
  \usepackage[T1]{fontenc}
  \usepackage[utf8]{inputenc}
  \usepackage{textcomp} % provide euro and other symbols
\else % if luatex or xetex
  \usepackage{unicode-math}
  \defaultfontfeatures{Scale=MatchLowercase}
  \defaultfontfeatures[\rmfamily]{Ligatures=TeX,Scale=1}
\fi
\usepackage{lmodern}
\ifPDFTeX\else  
    % xetex/luatex font selection
\fi
% Use upquote if available, for straight quotes in verbatim environments
\IfFileExists{upquote.sty}{\usepackage{upquote}}{}
\IfFileExists{microtype.sty}{% use microtype if available
  \usepackage[]{microtype}
  \UseMicrotypeSet[protrusion]{basicmath} % disable protrusion for tt fonts
}{}
\makeatletter
\@ifundefined{KOMAClassName}{% if non-KOMA class
  \IfFileExists{parskip.sty}{%
    \usepackage{parskip}
  }{% else
    \setlength{\parindent}{0pt}
    \setlength{\parskip}{6pt plus 2pt minus 1pt}}
}{% if KOMA class
  \KOMAoptions{parskip=half}}
\makeatother
\usepackage{xcolor}
\setlength{\emergencystretch}{3em} % prevent overfull lines
\setcounter{secnumdepth}{5}
% Make \paragraph and \subparagraph free-standing
\makeatletter
\ifx\paragraph\undefined\else
  \let\oldparagraph\paragraph
  \renewcommand{\paragraph}{
    \@ifstar
      \xxxParagraphStar
      \xxxParagraphNoStar
  }
  \newcommand{\xxxParagraphStar}[1]{\oldparagraph*{#1}\mbox{}}
  \newcommand{\xxxParagraphNoStar}[1]{\oldparagraph{#1}\mbox{}}
\fi
\ifx\subparagraph\undefined\else
  \let\oldsubparagraph\subparagraph
  \renewcommand{\subparagraph}{
    \@ifstar
      \xxxSubParagraphStar
      \xxxSubParagraphNoStar
  }
  \newcommand{\xxxSubParagraphStar}[1]{\oldsubparagraph*{#1}\mbox{}}
  \newcommand{\xxxSubParagraphNoStar}[1]{\oldsubparagraph{#1}\mbox{}}
\fi
\makeatother


\providecommand{\tightlist}{%
  \setlength{\itemsep}{0pt}\setlength{\parskip}{0pt}}\usepackage{longtable,booktabs,array}
\usepackage{calc} % for calculating minipage widths
% Correct order of tables after \paragraph or \subparagraph
\usepackage{etoolbox}
\makeatletter
\patchcmd\longtable{\par}{\if@noskipsec\mbox{}\fi\par}{}{}
\makeatother
% Allow footnotes in longtable head/foot
\IfFileExists{footnotehyper.sty}{\usepackage{footnotehyper}}{\usepackage{footnote}}
\makesavenoteenv{longtable}
\usepackage{graphicx}
\makeatletter
\def\maxwidth{\ifdim\Gin@nat@width>\linewidth\linewidth\else\Gin@nat@width\fi}
\def\maxheight{\ifdim\Gin@nat@height>\textheight\textheight\else\Gin@nat@height\fi}
\makeatother
% Scale images if necessary, so that they will not overflow the page
% margins by default, and it is still possible to overwrite the defaults
% using explicit options in \includegraphics[width, height, ...]{}
\setkeys{Gin}{width=\maxwidth,height=\maxheight,keepaspectratio}
% Set default figure placement to htbp
\makeatletter
\def\fps@figure{htbp}
\makeatother

\usepackage{region,hyperref,multirow}
\definecolor{mypink}{RGB}{219, 48, 122}
\makeatletter
\@ifpackageloaded{caption}{}{\usepackage{caption}}
\AtBeginDocument{%
\ifdefined\contentsname
  \renewcommand*\contentsname{Table of contents}
\else
  \newcommand\contentsname{Table of contents}
\fi
\ifdefined\listfigurename
  \renewcommand*\listfigurename{List of Figures}
\else
  \newcommand\listfigurename{List of Figures}
\fi
\ifdefined\listtablename
  \renewcommand*\listtablename{List of Tables}
\else
  \newcommand\listtablename{List of Tables}
\fi
\ifdefined\figurename
  \renewcommand*\figurename{Figure}
\else
  \newcommand\figurename{Figure}
\fi
\ifdefined\tablename
  \renewcommand*\tablename{Table}
\else
  \newcommand\tablename{Table}
\fi
}
\@ifpackageloaded{float}{}{\usepackage{float}}
\floatstyle{ruled}
\@ifundefined{c@chapter}{\newfloat{codelisting}{h}{lop}}{\newfloat{codelisting}{h}{lop}[chapter]}
\floatname{codelisting}{Listing}
\newcommand*\listoflistings{\listof{codelisting}{List of Listings}}
\makeatother
\makeatletter
\makeatother
\makeatletter
\@ifpackageloaded{caption}{}{\usepackage{caption}}
\@ifpackageloaded{subcaption}{}{\usepackage{subcaption}}
\makeatother

\ifLuaTeX
  \usepackage{selnolig}  % disable illegal ligatures
\fi
\usepackage[]{natbib}
\bibliographystyle{region}
\usepackage{bookmark}

\IfFileExists{xurl.sty}{\usepackage{xurl}}{} % add URL line breaks if available
\urlstyle{same} % disable monospaced font for URLs
\hypersetup{
  pdftitle={A practical guide to spatial interaction modelling},
  pdfauthor={Blinded},
  pdfkeywords={spatial interaction modelling, gravity modelling, flow
data},
  colorlinks=true,
  linkcolor={blue},
  filecolor={Maroon},
  citecolor={Blue},
  urlcolor={blue},
  pdfcreator={LaTeX via pandoc}}




\title{A practical guide to spatial interaction modelling}
\author{%
Blinded\parnote{}}
\date{}

\setcounter{page}{999}
\renewcommand{\thepage}{\arabic{page}}  % L for Letters, R for Resources, E for Editorial
\jvol{1} 
\jnum{1} 
\jyear{2020} 
\jpages{999--999} 
\jauthor{MISSING} 
\received{} 
\accepted{} 

\jdoi{10.18335/region.v??i??.???} 

\setlength{\parskip}{0pt plus1pt}
\setlength{\parindent}{15pt}

% %\bibliographystyle{plainnat}
\bibpunct{(}{)}{,}{a}{}{,}   % % changes formatting in natbib, see http://merkel.zoneo.net/Latex/natbib.php
% % It can be moved into the package call!

%%%%%%%%%%% GM inserted %%%%%%%%%%%%%%%
\usepackage[breakable]{tcolorbox}

% prompt
\makeatletter
\newcommand{\boxspacing}{\kern\kvtcb@left@rule\kern\kvtcb@boxsep}
\makeatother
\newcommand{\prompt}[4]{
	\ttfamily\llap{{\color{#2}[#3]:\hspace{3pt}#4}}\vspace{-\baselineskip}
}

\definecolor{incolor}{HTML}{303F9F}
\definecolor{outcolor}{HTML}{D84315}
\definecolor{cellborder}{HTML}{CFCFCF}
\definecolor{cellbackground}{HTML}{F7F7F7}
\definecolor{celloutborder}{HTML}{FFAFAF}
\definecolor{celloutbackground}{HTML}{F7E7E7}

\newcounter{code}
%%%%%%%%%%%%%%%%%%%%%%%%%%%%%%%%%%%%%%%

\newenvironment{ROutput}{\definecolor{shadecolor}{HTML}{F7E7E7}\begin{snugshade}}{\end{snugshade}}
\newenvironment*{RInput}{}{}
\begin{document}
\maketitle
\begin{abstract}
This document is only a demo explaining how to use the template.
\end{abstract}


\section{Introduction}\label{sec-intro}

Spatial interaction models (SIMs) is a core tool to simulate flows
between different locations in physical space. They are a valuable
resource through which the geographic structure between locations
encoded in aggregate flows of people, information and goods can be
represented and understood. Intuitively, SIMs seek to capture the
spatial interaction between places as a function of three components:
origin attributes, destination attributes and their separation. Inspired
by Newtonian concepts developed in physics, spatial flows between
locations are conceived as the result of their proportional
gravitational force and inverse association with spatial separation.
Attributes of origin and destination locations are employed to represent
gravitational forces pushing and pulling people, information and goods
between specific locations. Various forms of distance and costs are used
to represent the deterring effects of geographical separation on spatial
flows.

SIMs are widely used for prediction and inference. They are used to make
inference about the factors contributing to influence spatial flows.
They have been used to understand the magnitude and direction of
influence of individual and place-level attributes on geographic flows.
Understanding the effect of these factors offers valuable evidence to
inform the development of appropriate plans, strategies and
interventions \citep{fotheringham_okelly1989}. SIMs are also used to
make predictions of the size of spatial flows. These predictions are
normally used to assess the impact of interventions and creation of
``what-if'' scenarios, providing guidance for the identification of
optimal locations and size for potential new service units {[}REF{]}. In
this context, SIMs are often used to evaluate the impact of new bus
stops, shopping stores, schools or housing units on their potential
demand and traffic changes \citep{fotheringham_okelly1989}. To these
ends, SIMs have been used to address questions in a variety of settings,
including retail, migration, transport, trade, commuting, school travel
and more broadly urban planning.

Yet, the implementation of SIMs remains a challenge. Algorithms to
calibrate the parameters of SIMs have remained locked away, either
behind dense algebraic notation in dusty papers from the 1970s, or
behind paywalls of commercial software \citep{rowe2024}. Additionally,
\citet{rowe2024} noted a dearth of knowledge within geographical
education as SIMs are not widely taught in undergraduate programmes in
the same way as, for instance, regression models are taught in economics
or social psychology. This situation is argued to have occurred despite
the availability of effective routines to calibrate SIMs via popular
linear and general linear modelling frameworks, and as practical
expediency is sacrificed at the expense of theoretical or technical
prowess \citep{rowe2024}. The ways in which calibration procedures are
presented as lengthy mathematical derivation or passing reference to
ordinary least square tend to hamper accessibility for the easy
implementation of SIMs.

This computational notebook contributes to redressing these issues. It
aims to provide an intuitive, understandable and practical guide to
estimate SIMs in a variety of modelling frameworks. It will include the
necessary code to calibrate SIMs, using origin-destination
travel-to-work data for the United Kingdom in R programming language.
The code provided is generisable and can be adapted to different
origin-destination flow data and contexts, including migration, student,
transport, trade, currency, data transfer, vessel, shipment and freight
flows.

The notebook is structured as follows. The next section sets out some
fundamental concepts and definitions relating to SIMs.
Section~\ref{sec-comenv} identifies the libraries used before
Section~\ref{sec-data} describes the data. Section~\ref{sec-visualising}
illustrates key techniques to visualise complex spatial interaction
data, and Section~\ref{sec-modelling} shows and explains how to estimate
SIMs using a range of modelling frameworks. It start with traditional
mathematical and Ordinary Least Squares (OLS) approaches to more
advanced statistical frameworks, such as Generalised Linear Mixed Models
(GLMMs) and machine learning algorithms.

\section{Context}\label{context}

SIMs take various forms. Newtonian gravity models are probably the most
widely known and used form of SIMs. Inspired by Newton's law of gravity,
the basic gravity version of these models assumes that the spatial flows
or interactions between an origin (\(i\)) and a destination \(j\) is
proportional to their masses (\(M_{i}\) and \(M_{j}\)) and inversely
proportional to their separation (\(D_{ij}\)). Locations are expected to
interact in a positively reinforcing manner that is multiplicative of
their masses, but to diminish with the intervening role of their
separation. The separation between locations is often represented by a
distance decay function and is measured in terms of the distance, cost
or time involved in the interaction. Generally, the model includes a
constant (\(G\)) ensuring that the expected flows do not exceed their
respective observed counts, and a parameter (\(k\)) representing the
deterring effect of geographical separation. The task is to estimate
these parameters. Formally a gravity model can be expressed as:

\[
T_{ij} = G \frac{M_{i} M_{j}}{D^{k}_{ij} }
\]

SIMs have three key inputs: (1) a matrix of flows between a set of
origins and destinations; (2) a measure of separation between origins
and destinations; and, (3) measures of masses at origin and destination
locations. The literature usually considers a family of SIMs taking four
forms which refers to various constraints placed on parameters of the
model \citep{wilson1971}. There is an \emph{unconstrained} version which
is actually constrained to ensure that the total sum of the predicted
flows from a gravity model be equal the total sum of the observed flows
across all origins and destinations. Constrained versions are used to
ensure that specific origin or destination observations are met. Three
general formulations of constrained models are used:
\emph{production-constrained}, \emph{attraction-constrained} and
\emph{doubly-constrained} models. \emph{Production-constrained} versions
are used to constrain a model so that the predicted flows emanating from
individual origins is equal to their respective observed numbers.
\emph{Attraction-constrained} versions do the same but at individual
destinations. \emph{Doubly-constrained} versions combine these two sets
of constraints to ensure predicted flows are equal to observed flows at
both origins and destinations.

\section{Computation environment}\label{sec-comenv}

\section{Data}\label{sec-data}

Commuting or migration?

\section{Visualising spatial interaction data}\label{sec-visualising}

\section{Estimating spatial interaction models}\label{sec-modelling}

\subsection{Mathemathical gravity
models}\label{mathemathical-gravity-models}

\subsection{Statistical gravity
models}\label{statistical-gravity-models}

\subsection{Extensions}\label{extensions}

\subsubsection{Generalised linear mixed gravity
models}\label{generalised-linear-mixed-gravity-models}

\subsubsection{Machine learning gravity
models}\label{machine-learning-gravity-models}


\renewcommand\refname{References}
  \bibliography{bibliography.bib}



% This is the COPYRIGHT statement for the article.
\vspace*{\fill}
\tabcolsep0mm
\noindent
\begin{tabular*}{\textwidth}{ll}
	\toprule
	\multirow{2}{19mm}{\includegraphics[width=18mm,height=10mm]{_extensions/region-ersa/REGION/CC-BY-88x31}} & {\small \multirow{2}{328pt}{\textcopyright\  by the authors. Licensee: REGION -- The Journal of ERSA, European Regional Science Association, Louvain-la-Neuve, Belgium. This article is distri-}} \\
	& \\[-1pt]
	\multicolumn{2}{l}{\small \multirow{2}{\textwidth}{buted under the terms and conditions of the Creative Commons Attri\-bution (CC BY) license (\href{http://creativecommons.org/licenses/by/4.0/}{http://creativecommons.org/licenses/by/4.0/}).}}\\
	& \\
	\bottomrule
\end{tabular*}

\end{document}
